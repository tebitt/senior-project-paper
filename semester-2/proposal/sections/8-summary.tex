\section{Summary and Benefits}
\subsection{Summary}
This project centers on developing an emotional well-being robot designed to support mental health by helping users manage stress and anxiety. Using technologies such as computer vision and natural language processing, the robot can recognize and respond to emotional signals in real time, offering consistent and empathetic assistance. Its design emphasizes appearance, interactivity, and empathy, specifically catering to the needs of Gen Z and younger millennials in promoting mental wellness. The project is structured across two academic semesters, following an agile methodology to produce two major prototypes: MVP1 (Proof of Concept) and MVP2 (Fully Functional Design). The development is supported by insights from Dr. Paulo Fernando Rocha Garcia, Ph.D., Assistant Professor of AI and Robotics at Chulalongkorn University, and Ms. Kunpariya Siripanit, a counseling psychologist at Chulalongkorn University, ensuring that the robot meets both technical and mental health standards. Ultimately, this project provides an innovative, scalable solution that addresses the growing challenge of anxiety disorders, positioning the robot as a sustainable alternative to traditional mental health support.
\subsection{Benefits}
\begin{itemize}
    \item \textbf{Direct Industry Impact:} This project will greatly contribute to the mental health field, particularly addressing the needs of patients dealing with stress and general anxiety—an enormous market segment. Mental well-being robots are a crucial tool in helping manage “Terror Outbursts” in Thailand. By utilizing AI to detect and respond to emotions through facial recognition and voice analysis, these robots can reduce dependence on human intervention in promoting mental positivity. Moreover, they will improve the consistency and quality of mental health support, effectively addressing gaps in current care.
    \item \textbf{Scalability and Long-Term Value:} With global anxiety disorders rising by 55\% between 1990 and 2019, affecting an estimated 301 million people worldwide [39], these robots are poised to become increasingly vital. Their ability to provide real-time, personalized support will enhance individual well-being and contribute to the long-term sustainability of mental health care systems. Although the project is currently focused on proof of concept and prototyping, it offers a novel approach to mental well-being, with potential for future scalability and commercialization.
\end{itemize}
