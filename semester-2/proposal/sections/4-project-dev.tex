\section{Project Concept Development}

\subsection{Acknowledgement}
The project concept was developed under consultation with Ms. Kunpariya Siripanit, Counseling Psychologist at Chula Student Wellness (CUSW), Chulalongkorn University.

\subsection{Project Overview}
\textbf{Pookie}, the emotional well-being robot, is conceptualized as a responsive, AI-driven companion designed to improve mental well-being, specifically catered to customers under stress and anxiety. In this project, \textbf{anxiety} is indicated by signs of stress or worry, and \textbf{emotional well-being} is defined as \textit{"positivity promotion."}

The project focuses on \textbf{promotion}, meaning the promotion of positive well-being through the use of robotics, rather than \textbf{prevention}, which refers to the goal of preventing long-term issues such as depression or suicide.

The robot is envisioned to satisfy three key pillars of customer needs:
\begin{itemize}
    \item \textbf{Appearance}
    \item \textbf{Interactivity}
    \item \textbf{Empathy}
\end{itemize}
These will be achieved through the intuitive integration of computer vision, feature extraction, sensors, and actuators.

\subsection{Limitations and Scope}
This project aims to develop a proof of concept for an emotional well-being robot. However, several limitations and scope considerations apply:
\begin{enumerate}
    \item\textbf{Security:}
        \begin{itemize}
            \item The focus is on creating a prototype that demonstrates the feasibility of an emotional well-being robot.
            \item Security measures will be basic, and advanced features like data encryption and user authentication are beyond the project’s scope.
        \end{itemize}
    \item\textbf{Safety:}
        \begin{itemize}
            \item Fundamental safety, including electronics, heat output, and physical design, will be ensured through rigorous testing.
            \item Detailed safety protocols, such as long-term durability and fail-safes for unforeseen hazards, are outside the scope of this prototype phase.
        \end{itemize}
    \item\textbf{Functionality:}
        \begin{itemize}
            \item The robot will provide core emotional well-being functionalities such as basic interaction and mood assessment.
            \item Advanced features, like personalized therapeutic interventions or integration with external health systems, will not be included in this prototype.
        \end{itemize}
    \item\textbf{User Experience:}
        \begin{itemize}
            \item The prototype will offer a foundational user experience, but may lack the polish and customization of fully developed models.
            \item User interface enhancements will be addressed in future, scaled development phases.
        \end{itemize}
    \item\textbf{Scalability:}
        \begin{itemize}
            \item The project will not focus on scalability for mass production or widespread deployment.
            \item The prototype demonstrates initial concepts and feasibility, not full-scale implementation.
        \end{itemize}
    \item\textbf{Integration:}
        \begin{itemize}
            \item Extensive integration with other technologies or platforms is not within the project's focus.
            \item Emphasis will be on the standalone capabilities of the robot, with minimal focus on interoperability with existing systems.
        \end{itemize}
\end{enumerate}

By acknowledging these limitations and scope considerations, this project sets clear expectations for its initial development phase. Future iterations may address these areas in greater detail based on feedback and further research.
