\section{Project Outcome}

Over the course of the project, the team has had several deliverables. During the first semester, we successfully completed seven key deliverables: the project proposal, first and second progress reports, final report draft, final presentation, final report, and a live demo of MVP1. As outlined in the project plan, MVP1 served as a proof of concept, showcasing the basic emotion detection algorithms and their basic integration with the robot's hardware components.
Now, in the second semester and at the stage of MVP2, our focus has shifted to integration and expanding the robot's functionality. While the exact number of report deliverables is still being determined, we have already made significant progress toward developing a functional, testable prototype. The team is now engaged in extensive testing with our target customer segment to gather feedback that will guide further development and scale the robot for potential commercial use. However, it is important to note that the project’s final outcome will not be a commercially deployable product at this stage.
To summarize the project’s overarching objectives, the robot must fulfill three core pillars of customer expectations for emotional wellness robots: appearance, interactivity, and empathy. By the end of the project, we will have accomplished the following:
\begin{itemize}
\item Created an anthropomorphic outer shell that resonates with our target market, seamlessly integrating the necessary electronic components.
\item Implemented interactive features under the expert guidance of Chula Student Wellness (CUSW).
\item Developed an accurate emotion detection algorithm capable of effectively assessing the user’s anxiety state.
\item Fostered empathetic human-robot interactions that promote emotional wellness.
\item Validated our design through market research and user testing feedback.
\item Compiled a comprehensive report detailing user feedback from testing.
\end{itemize}
Quantitatively, by the end of the project, we expect the following outcomes:
\begin{itemize}
\item A measurable reduction in anxiety levels as reported by users during testing.
\item A measurable improvement in positivity levels as reported by users during testing.
\item Achieve a satisfactory sample size for testing.
\item Achieving benchmark accuracy across various emotion detection metrics, demonstrating a functional and effective model.
\end{itemize}
