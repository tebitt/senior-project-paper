\section{Testing Methodology}
\label{sec:testing-methodology}
This section proposes the methodology for end-user testing, designed to evaluate Pookie’s ability to recognize human emotions and deliver responses that promote emotional well-being—specifically by reducing anxiety and encouraging a positive mental state. Although the team initially planned to conduct these tests within the current semester, unforeseen time constraints prevented the completion of user trials.

The testing will take place in a controlled setting at the MI Innovation Labs, located in the 100th Year Engineering Building, over a period of 1 to 2 weeks. During this period, students will be invited to interact with Pookie and provide feedback on their experience. The MI Labs were selected due to their consistent foot traffic from our target demographic—users aged 18 and above—and their safe, enclosed environment conducive to focused interaction. A team member will be present at all times to supervise the sessions and administer pre- and post-interaction surveys, which will capture shifts in users' self-reported anxiety and positivity levels.

While this setup allows for a higher volume of interactions and supports more robust data analysis, it does not fully simulate Pookie’s envisioned integration into users’ daily routines.  One limitation of this approach is the testing location itself: the MI Innovation Labs primarily attract engineering students, which could lead to sampling bias. Nonetheless, as the target population includes individuals aged 18 and above with mild to moderate anxiety, engineering students still represent a reasonably valid subset of this broader demographic.

Each user interaction will last approximately five minutes, excluding time allocated for briefing and debriefing. A detailed breakdown of the experimental parameters is provided in Table \ref{tab:19-experiment}.

\begin{table}[h]
    \centering
    \begin{tabular}{|c|c|}
        \hline
        \textbf{Parameter} & \textbf{Detail} \\
        \hline
        Format   & Quantitative Questionnaire; Total 10 Questions   \\
        \hline
        Environment     & MI Innovation Labs, 100th Year Building, M Floor     \\
        \hline
        Experiment Time & Approximately 5 minutes per person     \\
        \hline
        Target Segment & Students aged 18 and above     \\
        \hline
    \end{tabular}
    \caption{Experiment Parameters}
    \label{tab:19-experiment}
\end{table}

As previously mentioned, users will be asked a series of questions before and after interacting with the robot, including some preliminary ones. For example, questions like "What is your occupation?" will help determine whether the robot has a more significant impact on specific user segments. Additionally, the numerical scores assigned will have specific interpretations, which we will later discuss with the psychology advisor (e.g., an anxiety score of 8 might indicate that the individual begins to struggle with tasks when anxious).

\subsubsection*{Preliminary Questions}
\begin{itemize}
    \item \textbf{Consensual:} Could you spare 5 minutes of your time for our experiment? We will collect data from you in the form of questionnaires, and we will only use your face and voice data temporarily for processing, where it will be completely erased afterwards, do you consent?
    \item \textbf{Demographic:} What is your age? What is your occupation? If you are a student, from which faculty are you studying?
\end{itemize}
\newpage
\subsubsection*{Questions asked before interaction}
\begin{itemize}
    \item \textbf{Psychographic:} On a scale from 1-10, how would you rate your everyday positivity? 
    \item \textbf{Psychographic:} On a scale from 1-10, how frequently do you feel positive/happy? 
    \item \textbf{Psychographic:} On a scale from 1-10, how would you rate your daily anxiety levels?
    \item \textbf{Psychographic:} On a scale from 1-10, how frequently do you feel anxious?
\end{itemize}


