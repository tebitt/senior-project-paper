\section{Project Outcome}
This section discusses the outcome of this project, aligning with objectives discussed in section \ref{sec:objective}. The section will cover all qualitative outcomes, but does not cover quantitative outcomes,  because user testing objectives could not be fulfilled in time. 

\subsection{Qualitative Outcomes}
The first qualitative objective was to \textit{"Design an intuitive appearance and interactive features for the robot under the expert guidance of Chula Student Wellness (CUSW)"}. To achieve this, the team conducted interviews with psychologists from CUSW to gather professional insights into effective design principles for therapeutic robots. Key recommendations emphasized a neutral, pet-like aesthetic and the integration of interactive and empathetic features to foster emotional connection and comfort. Guided by these insights, the team engaged in multiple design iterations using Fusion360, followed by prototyping through 3D printing. 

The second objective was to \textit{“Develop an accurate emotion detection algorithm that captures the user’s emotional state, leveraging software design principles taught throughout the curriculum”}. As discussed in section \ref{sec:classification-layer} and Tables \ref{tab:8-test} and \ref{tab:10-ser}, the models that the team developed or leveraged achieved satisfactory metrics in accuracy. These models, complemented with Bayesian networks and decision trees, constitute a robust emotion detection pipeline which can be scaled further should newer technologies achieve even higher benchmark metrics. Key software design principles taught in the curriculum that were leveraged include those from \textit{parallel programming, AI model development and mechatronics software}.

The third objective was to \textit{“develop empathetic human-robot interactions that promote emotional wellness within the customer, leveraging various engineering design principles taught throughout the curriculum”}. Through intensive design and prototyping, the team singled down on the user experience design shown in Section \ref{sec:ux}. To successfully carry out this design, multiple engineering principles such as \textit{mechanical engineering design} were incorporated, fulfilling the objective.

The last objective was to \textit{“Conduct extensive testing and refinement based on user feedback”}. This objective, unfortunately, was not fulfilled in time, where the team initially intended to carry out the testing methodology discussed in Section \ref{sec:testing-methodology}. 
