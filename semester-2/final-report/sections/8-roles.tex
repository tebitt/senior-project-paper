\section{Roles and Responsibilities}
Although each student is assigned to a specific component of the project, collaboration remains a core value, and responsibilities may evolve over time to support the overall success of the project.

\subsection*{Kridbhume Chammanard – Project Manager}
 As the Project Manager, Kridbhume Chammanard is responsible for overseeing the entire project and ensuring all activities remain aligned with established goals and deadlines. Kridbhume delegates tasks to the project engineers, manages the project timeline, and adapts workflows as necessary to maintain progress. A critical aspect of the role involves regularly engaging with advisors from ISE and Chula Student Wellness, providing updates and seeking feedback. Kridbhume also reviews and approves all deliverables before submission, ensuring quality and consistency. Additionally, Kridbhume is responsible for managing the overall budget, ensuring that expenditures remain within limits and that resources are allocated effectively across the team.

 \subsection*{Thitaya Divari – Project Engineer, Product Development}
 Thitaya Divari leads the product development efforts, with a focus on designing and refining the user experience of the robot. Using CAD software, Thitaya develops detailed models and interfaces, ensuring the final design is user-friendly and functional. Thitaya is responsible for sourcing all necessary hardware components and overseeing their assembly and testing to ensure alignment with design specifications. Managing the hardware and product development budget is also part of the role, which includes tracking expenses and adjusting plans to stay within financial constraints. Thitaya also maintains comprehensive documentation of the development process, including design specifications, component inventories, and testing results.

\subsection*{Tibet Buramarn – Project Engineer, Software Development}
 Tibet Buramarn is responsible for the software development of the robot, including implementation, testing, and ongoing refinement. This role involves designing the software architecture, writing and debugging code, and ensuring seamless integration between software and hardware systems. Tibet also manages DevOps practices to facilitate efficient deployment and maintenance of the robot’s functionalities. A key aspect of Tibet’s work includes identifying and resolving compatibility or performance issues and optimizing system responsiveness. Through extensive testing, Tibet ensures that the software is reliable, user-friendly, and aligned with the project’s core objectives.
