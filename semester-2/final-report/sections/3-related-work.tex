\section{Related Work}
This section explores past initiatives involving emotional wellness and support robots. In summary, these robots integrate hardware features with advanced technologies such as artificial intelligence to facilitate personalized and empathetic interactions with users. By tailoring their responses to individual emotional states, these systems aim to reduce stress and anxiety, ultimately promoting emotional well-being.

Firstly, Moxie is an emotional support robot designed for children, especially those needing help with anxiety or emotional development. Equipped with AI and machine learning, Moxie interacts empathetically to help children build social, emotional, and cognitive skills. It engages in personalized conversations, encourages emotional expression, and uses storytelling, breathing exercises, and interactive games to foster learning. Moxie also tracks progress and adapts to each child’s needs, providing daily activities and challenges that promote emotional growth \cite{hurst2020socialemotionalskillstraining}.

Next, LOVOT, developed by Groove X, is a social robot designed to provide companionship and emotional support, aiming to alleviate loneliness and enhance emotional well-being through interactive and engaging experiences. With its large, expressive eyes that blink and change expressions, LOVOT establishes emotional connections by making eye contact and responding to users. It engages in playful interactions, such as following users or reacting to touch with affectionate movements, creating a sense of care and attachment \cite{lovot}. 

Lastly, PARO, a robotic baby seal developed by the National Institute of Advanced Industrial Science and Technology (AIST) in Japan, is one of the most well-known emotional support robots. It is designed to provide companionship to elderly individuals, particularly those with dementia or cognitive impairments. PARO incorporates several advanced features to enhance its therapeutic effectiveness: it is equipped with tactile sensors that allow it to respond to touch, microphones to recognize voices and sounds, and artificial intelligence to learn and adapt to the user’s preferences over time. It also has movement and behavior patterns that mimic a real seal, such as blinking its eyes, moving its flippers, and making sounds, which helps in creating a comforting and engaging interaction. Studies have shown that PARO can reduce stress and anxiety, foster socialization, and enhance overall mood among users \cite{wada2006}.
