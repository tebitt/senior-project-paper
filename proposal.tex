\documentclass[a4paper,10pt]{article}

\usepackage{graphicx}
\graphicspath{{sections/images/}}

\usepackage{amsmath}
\usepackage{amsfonts}
\usepackage{amssymb}
\usepackage{hyperref}
\usepackage[a4paper, top=2cm, bottom=2cm, left=2.5cm, right=2.5cm]{geometry}

\begin{document}
\begin{titlepage}
    \centering  
    \begin{figure}[ht]
        \centering
        \includegraphics[width=\textwidth]{ise-logo.png}
    \end{figure}
    \vspace*{2cm} 
    
    {\Huge \textbf{Project Proposal:} \textit{Emotional Wellness Robot} \par}
    \vspace{4cm}
    
    {\large \textbf{Authors:} Tibet Buramarn, Kridbhume Chammanard, and Thitaya Divari \par}
    \vspace{1cm}
    {\large \textbf{Advisor:} Dr. Paulo Fernando Rocha Garcia and \par}
    {\large Ms. Kunpariya Siripanit \par}

    \vspace{3cm}
    
    {\large 2147416 Final Project I \par}
    {\large International School of Engineering (ISE) \par}
    {\large Chulalongkorn University \par}
    
    \vspace{2cm}
    
    {\large September 6, 2024 \par}
    
    \vspace*{\fill}
\end{titlepage}

\thispagestyle{empty}

\newpage
\begin{center}
        \item\section*{Abstract}
\end{center}
\large
Pookie, an AI-driven robot for promoting mental wellbeing, developed under advisory with Chula Student Wellness, aims to create an AI-driven companion to enhance mental well-being by promoting positivity. The robot aims to address “Terror Outbursts”, a future concern in Thailand involving an anxiety driven society, where the robot aims to alleviate feelings of stress and anxiety by providing a feeling of slowness and emotional attachment. The robot integrates computer vision, feature extraction, sensors, and actuators to address key customer needs in appearance, interactivity, and empathy. The general appearance features an anthropomorphic form with LED displays and sensors for interaction, drawing inspiration from existing robots like Kiki and Eilik, with future iterations expected to refine user experience and functionality based on feedback.  
\newpage
\begin{center}
        \item\section*{Acknowledgements}
\end{center}
The team expresses their sincere gratitude to Dr. Paulo Fernando Rocha Garcia, Ph.D., Assistant Professor of AI and Robotics at Chulalongkorn University, and Ms. Kunpariya Siripanit, Counseling Psychologist at Chula Student Wellness, Chulalongkorn University. Dr. Paulo’s expertise in artificial intelligence and robotics was instrumental in guiding the technical development of the emotional wellness robot, while Ms. Kunpariya’s insights into counseling psychology ensured the project’s alignment with mental health principles. The team also acknowledges the support provided by the faculty and staff of the International School of Engineering, whose assistance was invaluable throughout this project.
\normalsize

\newpage
\tableofcontents

\newpage
\begin{center}
\section*{Pookie: An AI-Driven Robot for Promoting Mental Wellbeing and Emotional Support}
\label{sec:topic}


\end{center}
\section{Research Background}
This section will provide justifications for the project and necessary knowledge for the reader in order to understand technical terms used throughout the report.
\subsection{Justification of the Project}

The project focuses on developing emotional wellbeing robots that promote mental wellness and positivity for users under the influence of stress and anxiety. In general, emotion-related robots are designed to respond to human emotions and can potentially achieve clinical outcomes similar to traditional therapy \cite{palmer2024}. Research has shown that digital interventions, such as AI-powered mental wellbeing robots, can effectively reduce anxiety symptoms and address unmet mental health needs, offering a promising solution to supplement traditional therapeutic approaches \cite{mamatha2024}.

The mental health industry faces significant challenges that cannot be fully addressed through human intervention alone \cite{charles2024}. Key issues include loneliness and social isolation, which are major contributors to depression, anxiety, and overall deterioration in mental health \cite{goh2023}, as well as therapeutic challenges where patients with dementia or other cognitive impairments often struggle with traditional therapeutic activities \cite{sukhawathanakul2021}.

In a study by the National Innovation Agency (NIA) based in Thailand, it was identified that the concept of "Terror Outburst" would become a pressing issue in Thailand by the year 2033 \cite{nia2023}. To elaborate, terror outburst refers to a society driven by constant fear and anxiety. Given this, traditional methods of addressing anxiety, such as therapy and medication, may not be accessible or appealing to everyone. This creates a significant pain point for individuals seeking immediate, non-invasive support. Our target customer segment includes young adults and professionals aged 18-35 who experience mild to moderate anxiety but may be hesitant to seek conventional treatment, where we provide an innovative alternative to support their mental wellbeing.

To ensure emotional wellbeing robots meet user needs and deliver effective support, three key pillars are essential: appearance, interactivity, and empathy. First, the robot's appearance should strike a balance between human-like and machine-like traits, fostering both comfort and trust in users \cite{decet2024}. High interactivity is also crucial; the robot should provide adaptive feedback through various stimuli to engage users effectively and enhance their emotional states \cite{wang2024}. Moreover, a robot's perceived empathic abilities play a significant role in how users interact with it, as these perceptions directly influence their willingness to attribute mental states to the robot, thereby impacting the overall quality of the interaction \cite{lillo2024}.

On the concept of emotion detection, traditional emotional detection methods utilize verbal and non-verbal cues to accurately detect and respond to human emotions. Verbal cues like pitch variations, volume \cite{hakanpaa2021}, and speech rate \cite{wang2018} are critical indicators of emotional states. For example, higher pitch and increased volume often signal heightened emotional arousal \cite{rodero2011}, as seen in both American English and Mandarin Chinese, where pitch and speed are essential for expressing emotions. Additionally, contextual understanding—interpreting emotions based on situational cues—further refines the robot's emotional recognition capabilities \cite{abbas2023}. Non-verbal cues, such as facial expressions and body language, also play a vital role. For instance, a smile usually denotes happiness \cite{chapre2023}, while crossed arms might suggest defensiveness \cite{liu2024}. By integrating these verbal and non-verbal indicators, mental wellbeing support robots can offer tailored responses, thereby improving the overall effectiveness of their interactions with users.

\subsection{Necessary Knowledge}

The development of an mental wellbeing support robot with emotional detection capabilities requires a strong foundation in various advanced concepts within artificial intelligence, machine learning, robotics, and human-computer interaction. Below is an overview of the essential knowledge areas for this project:

Machine learning models are the backbone of emotion detection systems. Convolutional Neural Networks (CNNs) \cite{taye2023} are widely used for tasks such as facial emotion recognition, where they excel at analyzing image data to identify patterns corresponding to different emotional states. In contrast, Recurrent Neural Networks (RNNs) and Long Short-Term Memory (LSTM) networks \cite{schmidt2019} are essential for processing sequential data, such as audio signals, enabling the detection of emotions through vocal features.

Effective emotion detection relies on extracting meaningful features from raw data. For instance, Mel-Frequency Cepstral Coefficients (MFCCs) \cite{singh2014} are a crucial feature extraction technique in speech emotion recognition, capturing the essential characteristics of the audio signal that correlate with emotional states. In the visual domain, key facial features like eyes, mouth, and eyebrows are extracted and analyzed by CNNs to detect emotions from facial expressions.

Natural Language Processing (NLP) is crucial for enabling the robot to understand and interpret human language, which is key to detecting emotions from text or spoken input. NLP techniques allow the robot to process language data, extracting meaningful insights such as sentiment and intent. These insights help the robot assess the emotional tone and context of the user's communication, making it possible to respond appropriately to their emotional needs \cite{khurana2017}.

The design of emotionally intelligent robots requires an understanding of Human-Robot Interaction (HRI) principles. These principles guide the development of robots that can interact naturally and empathetically with humans. Concepts such as user-friendly interface design, adaptive behavior, and empathetic response mechanisms ensure that the robot's interactions are socially acceptable and supportive \cite{bartneck2024}.

Additionally, emotionally intelligent robots rely on a combination of advanced hardware and software to accurately detect and respond to human emotions. Key hardware components, including cameras, are essential for capturing detailed facial expressions in real-time, allowing systems to effectively analyze emotional states \cite{gupta2024}. Microphones and audio sensors play a crucial role in gathering vocal cues, which are vital for emotion detection \cite{rastogi2023}. Processors and GPUs manage the heavy computational tasks, while actuators and motors control the robot's physical movements, such as gestures and facial expressions, enabling the robot to convey empathy and respond to users effectively. Haptic sensors further enhance this interaction by reacting to touch, contributing to a more interactive and supportive user experience.
\newpage
\section{Objective}

\subsection{Main Objective Statement}
The primary objective of this project is to design, develop, and deploy an emotional wellness robot capable of recognizing and responding to stress and anxiety symptoms in users through the integration of AI technologies such as computer vision and natural language processing. The robot must fulfill all three key pillars of customer expectations in emotional wellness robots: design, interactivity, and empathy.

\subsection{Specific Goals}
\begin{itemize}
    \item Design an anthropomorphic robotic outer shell that resonates with the target customer segment.
    \item Design interactive verbal and non-verbal features for the robot, such as making sounds when the robot is touched on the head, with expert supervision from Chula Student Wellness (CUSW).
    \item Develop an accurate emotion detection algorithm that captures the user’s state of anxiety through speech emotion recognition and facial expression recognition.
    \item Develop empathetic human-robot interactions that promote emotional wellness within the customer.
    \item Finalize a customer-centric design obtained through market validation efforts.
    \item Conduct extensive testing and refinement based on user feedback.
\end{itemize}

\subsection{Measurable Outcomes}
\begin{itemize}
    \item Achieve a relative reduction in self-reported anxiety.
    \item Achieve a benchmark in accuracy metrics for emotion detection.
    \item Obtain an improved before-and-after positivity score.
\end{itemize}

\subsection{Relevance or Significance}
With “Terror Outbursts” being one of the major societal challenges in Thailand, there is a pressing need for accessible positivity promotion. Our robot aims to bridge the gap between traditional therapy sessions by providing immediate support to individuals struggling with anxiety.
\newpage
\section{Literature Survey and Review}
\label{sec:literature}
This section will review existing literature related to the project. Discuss previous studies, findings, and theories that are relevant to your research. Highlight gaps in the literature that your project aims to fill.

\newpage
\section{Project Concept Development}

\subsection{Acknowledgement}
Project concept was developed under consultation with Ms. Kunpariya Siripanit, Counseling Psychologist at Chula Student Wellness (CUSW), Chulalongkorn University.

\subsection{Project Overview}
Pookie, the emotional wellbeing robot, is conceptualized as a responsive, AI-driven companion designed to provide  improve mental well-being, specifically catered to customers under the influence of stress and anxiety. To be specific, anxiety in this project is indicated by signs of stress or worry, and the definition of “emotional wellbeing” is “positivity promotion”. It is important to note that the scope of the project, in terms of emotional support, is identified as “Promotion”, meaning promotion of positive wellbeing through the use of robotics, rather than “Prevention”, which refers to a specific goal of preventing long term issues such as depression or suicide.

As mentioned, the project envisions a robot that satisfies three key pillars of customer needs: appearance, interactivity, and empathy. This will be accomplished by intuitive integration of computer vision, feature extraction, sensors and actuators.

\subsection{Limitations and Scope}
This project aims to develop a proof of concept for an emotional well-being robot, drawing inspiration from existing models such as Kiki and Eilik. However, there are several limitations and scope considerations for this project:

\textbf{Security:} The primary focus of this project is to create a prototype that demonstrates the feasibility of an emotional well-being robot. As such, the security measures implemented will be at a basic level. Comprehensive security features, including data encryption and advanced user authentication, are beyond the scope of this project.

\textbf{Safety:} While the robot will undergo rigorous testing to ensure fundamental safety in terms of electronics, heat output, and physical design, the scope of safety considerations will be limited to these basic aspects. Detailed safety protocols, including long-term durability and fail-safes for unforeseen hazards, will not be extensively addressed in this prototype phase.

\textbf{Functionality:} The robot will focus on core emotional well-being functionalities, such as basic interaction and mood assessment. Advanced features, such as personalized therapeutic interventions or integration with external health systems, will not be included in this prototype.

\textbf{User Experience:} The prototype will provide a foundational user experience but may lack the polish and customization of fully developed models. User interface and experience enhancements will be considered in future, scaled development phases.

\textbf{Scalability:} The project will not address scalability concerns for mass production or widespread deployment. The prototype is intended to demonstrate initial concepts and feasibility rather than full-scale implementation.

\textbf{Integration:} This project will not explore extensive integration with other technologies or platforms. The focus will remain on the standalone capabilities of the robot, with minimal emphasis on interoperability with existing systems.

By acknowledging these limitations and scope considerations, this project sets clear expectations and defines the boundaries of its initial development phase. Future iterations may address these areas in greater detail based on feedback and further research.


\newpage
\section{Project Planning and Timeline}

The overall project will span a total of two academic semesters of the senior year (a total of approximately 8 months) and will comprise a set of goals for each semester. The project will be managed using an agile methodology, where by the end of the project, two deliverables will be obtained: MVP1 and MVP2. This section will break down the project planning and timeline for each MVP, as well as expected deliverables for each phase.

\subsection{Channels}

Throughout the project, two essential tools will be used to facilitate communication and task delegation within the project. The first tool is Discord, a multi-functional communication tool that is practical for meetings, scheduling events, and so on. Discord will be used as the primary communication tool for the members in the project, as well as for some advisors. The second tool is Jira, an agile project management tool that facilitates task delegation and software project management. Jira will be used to track the tasks of each member in the project, as well as to track software features and bugs within the project in the form of tickets for ease of audit. Additionally, it will also comprise the customer journey of each feature of the robot in the form of “user stories.”

\subsection{MVP 1 - Proof of Concept \& Customer-Centric Specifications}

MVP 1 will span the entirety of academic semester 1 (from September until December) and will focus on delivering a proof of concept of the project, as well as feature specifications that focus on the customers’ needs. MVP 1 will comprise three sprints, each lasting for around a month.

The project starts at MVP 1 Sprint 0, which focuses on preliminary research and feature definition. This sprint will span the entirety of September, and comprises the project proposal, in-depth customer journey, and technology specifications (such as specifically which AI models to be used). MVP 1 Sprint 0 will have two deliverables: the written project proposal and first progress report.

In the next phase, MVP 1 Sprint 1, which lasts from October to early November, we will focus on preliminary software development aimed towards providing a proof of concept for the emotion detection algorithm. For each AI model that will be used, we will allocate time for data collection, training, and iterative quality checks using real data. In general, emotion detection will comprise detection of facial expressions, Speech Emotion Recognition (SER), and Gesture Recognition technologies, utilizing computer vision and feature extraction technologies. In addition, there are plans to integrate context understanding into the robot using NLP technologies, but we will not include it in the scope. Additionally, the UX design will also be drafted, consisting of a complete customer journey for each feature, as well as a basic outer shell for the robot that fits the design specifications. Sprint 1 will have three deliverables: second progress report, third progress report, and completed prototypes for emotion detection.

In the last phase, MVP 1 Sprint 2, which lasts from November to early December, we will focus on acquiring the first batch of hardware components for the project and testing the emotion detection models on our selected microcontroller. The microcontroller must have enough compute power to perform inferences in real time. Additionally, Sprint 2 will involve market validation, which comprises testing on the customer segment to obtain feedback for improvement in the following semester. By the end of MVP 1, we expect a prototype that integrates software and hardware components on a feasible level.

\begin{figure}[ht]
    \centering
    \includegraphics[width=\textwidth]{gantt-table.png}
    \caption{Gantt Chart for MVP 1 Timeline}
    \label{fig:gantt}
\end{figure}
\newpage
\subsection*{Task Number:}
\begin{enumerate}
\item\textbf{MVP 1 - Sprint 0}\\
1.1	Advisor Onboarding\\
1.2	Project Scoping\\
1.3	Project Scoping - Expert Interviews\\
1.4	Project Proposal\\
1.5	In-Depth Technology Specifications\\
1.6	E2E Customer Journey\\
\item\textbf{MVP 1 - Sprint 1}\\
2.1	Facial Expression Model - Data Collection\\
2.2	Facial Expression Model - Training/Testing\\
2.3	Facial Expression Model - Quality Testing\\
2.4	SER Model - Data Collection\\
2.5	SER Model - Training/Testing\\
2.6	SER Model - Quality Testing\\
2.7	Gesture Model - Data Collection\\
2.8	Gesture Model - Training/Testing\\
2.9	Gesture Model - Quality Testing\\
2.10 UX Design (Fusion360)\\
\item\textbf{MVP 1 - Sprint 2}\\
3.1	Parts Procurement\\
3.2	Jetson Orin/Nano Test\\
3.3	MVP 1 Integration \\
3.4	Market Validation\\
3.5	Final Presentation Preparations\\
\end{enumerate}

\newpage
\subsection{MVP 2 - Non-Commercial Prototype}

In this project proposal, the specific details of MVP 2 will not be disclosed. However, in general, MVP 2 will comprise three sprints, similar to MVP 1, but will focus on integration of the emotion detection algorithm with hardware components to create an output. MVP 2 will last the entirety of academic semester 2, and is expected to deliver a fully functional prototype, but will not be implemented to the extent of commercialization. Additionally, MVP 2 will focus on user testing, involving measurable metrics mentioned in the objectives section. The prototype will not encompass full consideration of security, safety, and real usage, but will focus on fulfilling the customer demands of appearance, interactivity, and empathy.

\subsection{Resource Allocation}

The project team comprises three senior engineering students. The roles for the project will be designated as follows: project manager, project engineer for product development, and project engineer for software development. The project manager is responsible for overseeing the entire project, delegating tasks, managing the timeline, and creating engagements with advisors from ISE and Chula Student Wellness. They are expected to streamline work processes for the project engineers and professionally check deliverables before submission. The project engineers are separated into two categories: product development and software development. The product development engineer will be responsible for UX design through CAD software, gathering hardware components, and managing the budget for doing so. The software development engineer will be responsible for overseeing all software initiatives, as well as managing DevOps practices within the project.

\subsection{Budget Allocation}

While the exact hardware components for the robot cannot be deduced yet, the table below will illustrate an approximate allocation of the budget provided for the project. Note that the budget specified is the maximum that was allocated for expenditure. The budget for each component may vary, but should never exceed the maximum allocated budget.

\begin{table}[ht]
\centering
\begin{tabular}{|l|l|}
\hline
\textbf{Component} & \textbf{Maximum Allocated Budget (THB)} \\ \hline
Microcontroller & 30,000 \\ \hline
Camera & 5,000 \\ \hline
Microphone & 5,000 \\ \hline
Electronics & 15,000 \\ \hline
Chassis and Framework & 5,000 \\ \hline
Decorative Components & 5,000 \\ \hline
LED Display & 5,000 \\ \hline
Miscellaneous & 5,000 \\ \hline
\textbf{Maximum Expenditure} & \textbf{75,000} \\ \hline
\end{tabular}
\caption{Approximate Budget Allocation for Hardware Components}
\end{table}

\newpage
\section{Theory and Technical Backup}

\subsection{Hardware Features}

The physical design of the robot is anthropomorphic-centric, with elements of animals as well. The robot will be around 12 inches tall, comprising various integrated hardware components. Starting from the top, the robot’s head will consist of a 3D printed sphere, and eyes made from an integrated LED display, which will be used as a form of interaction with the user. In addition, the head will also house the camera and microphone, used to receive image and sound inputs for processing within the microcontroller. Next, the robot’s body will consist of a large chassis to house the electrical components and the microcontroller. The body will also consist of motors located on the arms to allow for minor arm movement. Lastly, touch sensors will be placed in certain parts of the robot to imitate petting interaction. This design is akin to many desktop companion robots for promoting mental wellness, such as Kiki or Eilik, with the core difference being in the integration of various emotion detection methods.

\subsection{Software Features - Overview}

The robot comprises two main features: emotion detection and interaction. Emotion detection is an initiative to incorporate empathy for the customer experience with the robot, using computer vision to analyze facial expressions, as well as speech emotion recognition to analyze tone and pitch. Given predicted emotional status, the robot will be programmed to provide interaction in two forms: verbal and non-verbal. Verbal interactions consist of noises made by the robot, whereas non-verbal interactions comprise physical actions from the robot such as arm movement or changes in the LED display resembling its eyes.

\subsection{Emotion Detection Model - Facial Expression Recognition}

Facial expression recognition is a core technique in emotion detection systems, crucial for understanding non-verbal emotional cues in humans. Recent advancements have centered on using Convolutional Neural Networks (CNNs) to detect and classify facial emotions with high accuracy. OpenFace \cite{baltrusaitis2018} and VGGFace \cite{zhang2023} are among the most prominent models in this field. These models extract key facial features—such as eye movement, mouth shape, and eyebrow positions—from images and videos to classify emotions like happiness, anger, and sadness. CNNs have demonstrated strong performance by learning spatial hierarchies of features, allowing them to detect subtle changes in facial expressions, even in complex or dynamic environments. Such models are pivotal in creating emotionally responsive robots, as they allow real-time emotion tracking through visual input.

\subsection{Emotion Detection Model - Speech Recognition}

Speech recognition for emotional detection focuses on analyzing vocal characteristics to infer emotional states. Key prosodic features such as pitch, tone, and rhythm are crucial indicators of emotions in speech. Recent methods have employed Recurrent Neural Networks (RNNs) and Long Short-Term Memory (LSTM) networks for processing audio data, while newer approaches utilize transformer models like BERT \cite{devlin2019}. These models capture the temporal dependencies in spoken language, enabling the system to interpret emotions more accurately. LSTMs are particularly effective at maintaining contextual information over time, which is vital for understanding emotions that are expressed through vocal modulations. The integration of these techniques allows robots to engage in emotionally intelligent conversations by understanding the user's mood through their voice.
\newpage
\subsection{Testing}

The testing of the robot will comprise both internal and external components. In terms of internal components: The emotion detection model should achieve a benchmark score in accuracy metrics, or a similar number on a different scale. Since both the computer vision and speech emotion recognition (SER) datasets will be discretely labeled, quantitative measurement of accuracy will be feasible. For instance, the speech recognition model may classify a person’s voice as either normal or anxious.

On the other hand, in terms of external components, user testing will involve supervision and validation from Chula Student Wellness. Questionnaires will be developed to obtain the required information accordingly, representative questions include:

\begin{itemize}
    \item \textbf{Self-Report on Anxiety}
    \begin{enumerate}
        \item Before using Pookie, how would you rate your anxiety levels from 1-10? Please provide your rating now and when you’re at your most anxious.
        \item After using Pookie, how would you rate your anxiety levels from 1-10? Please provide your rating now and when you’re at your most anxious.
    \end{enumerate}
    \item \textbf{Positivity Promotion}
    \begin{enumerate}
        \item Before using Pookie, how would you rate your general positive feelings in life from 1-10?
        \item After using Pookie, how would you rate your general positive feelings in life from 1-10?
    \end{enumerate}
    \item \textbf{Attachment}
    \begin{enumerate}
        \item After using Pookie for 1 week, do you feel more attached to it?feelings in life from 1-10?
    \end{enumerate}
\end{itemize}

While these are examples of questions that will be asked, proper guidance from Chula Student Wellness will be given as well during the testing stage.
\newpage
\section{Project Outcome}
\label{sec:outcome}
This section should describe the expected outcomes of the project. What deliverables will be produced? How will the results be measured and evaluated?

\newpage
\section{Summary and Benefit to Real Industry}
\label{sec:summary}
Summarize the proposal and emphasize the potential impact of the project on the industry. Discuss how the research can address real-world problems and contribute to industry advancement.

\newpage
\section{Workload or Responsibilities}
\label{sec:workload}
Outline the distribution of workload among the project team members. Specify the responsibilities assigned to each team member and their expected contribution to the project.


\newpage
\addcontentsline{toc}{section}{References}
\bibliographystyle{IEEEtran}
\bibliography{sections/references}

\end{document}
