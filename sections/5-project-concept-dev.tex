\section{Project Concept Development}

\subsection{Acknowledgement}
Project concept was developed under consultation with Ms. Kunpariya Siripanit, Counseling Psychologist at Chula Student Wellness (CUSW), Chulalongkorn University.

\subsection{Project Overview}
Pookie, the emotional wellbeing robot, is conceptualized as a responsive, AI-driven companion designed to provide  improve mental well-being, specifically catered to customers under the influence of stress and anxiety. To be specific, anxiety in this project is indicated by signs of stress or worry, and the definition of “emotional wellbeing” is “positivity promotion”. It is important to note that the scope of the project, in terms of emotional support, is identified as “Promotion”, meaning promotion of positive wellbeing through the use of robotics, rather than “Prevention”, which refers to a specific goal of preventing long term issues such as depression or suicide.

As mentioned, the project envisions a robot that satisfies three key pillars of customer needs: appearance, interactivity, and empathy. This will be accomplished by intuitive integration of computer vision, feature extraction, sensors and actuators.

\subsection{Limitations and Scope}
This project aims to develop a proof of concept for an emotional well-being robot, drawing inspiration from existing models such as Kiki and Eilik. However, there are several limitations and scope considerations for this project:

\textbf{Security:} The primary focus of this project is to create a prototype that demonstrates the feasibility of an emotional well-being robot. As such, the security measures implemented will be at a basic level. Comprehensive security features, including data encryption and advanced user authentication, are beyond the scope of this project.

\textbf{Safety:} While the robot will undergo rigorous testing to ensure fundamental safety in terms of electronics, heat output, and physical design, the scope of safety considerations will be limited to these basic aspects. Detailed safety protocols, including long-term durability and fail-safes for unforeseen hazards, will not be extensively addressed in this prototype phase.

\textbf{Functionality:} The robot will focus on core emotional well-being functionalities, such as basic interaction and mood assessment. Advanced features, such as personalized therapeutic interventions or integration with external health systems, will not be included in this prototype.

\textbf{User Experience:} The prototype will provide a foundational user experience but may lack the polish and customization of fully developed models. User interface and experience enhancements will be considered in future, scaled development phases.

\textbf{Scalability:} The project will not address scalability concerns for mass production or widespread deployment. The prototype is intended to demonstrate initial concepts and feasibility rather than full-scale implementation.

\textbf{Integration:} This project will not explore extensive integration with other technologies or platforms. The focus will remain on the standalone capabilities of the robot, with minimal emphasis on interoperability with existing systems.

By acknowledging these limitations and scope considerations, this project sets clear expectations and defines the boundaries of its initial development phase. Future iterations may address these areas in greater detail based on feedback and further research.

