\section{Project Outcome}

Over the span of the project, the team will have many deliverables. During the first semester, there will be a total of seven deliverables: project proposal, first progress report, second progress report, final report draft, final presentation, final report, and a live demo of MVP1. As mentioned in project planning, MVP1 is expected to be a proof of concept of the basic emotion detection algorithms and their integration with other hardware components.

On the other hand, during the second semester, the number of deliverables in the form of reports has not been disclosed yet, but a functional, testable prototype will be completed. The team expects to perform intensive testing through the customer segment to obtain feedback for scaling the robot for commercial use. However, the outcome of the project does not envision a robot ready for commercial deployment.

To recap the overarching goals of the project, the robot must fulfill three key pillars of customer expectations for emotional wellness robots: appearance, interactivity, and empathy. By the end of the project, we will have completed:

\begin{itemize}
    \item An anthropomorphic robotic outer shell that resonates with the target customer segment, which seamlessly houses and integrates electronic components.
    \item Interactive and feasible features for the robot through elements such as touch, under expert supervision from Chula Student Wellness (CUSW) 
    \item An accurate emotion detection algorithm that effectively captures the user’s anxiety state.
    \item Empathetic human-robot interactions that promote emotional wellness.
    \item A customer-centric design through market validation efforts.
    \item A thorough report on user feedback after testing the robot.
\end{itemize}

Quantitatively, by the end of the project, we expect the following metrics:

\begin{itemize}
    \item A relative reduction in anxiety levels from user testing which will be measured as a self-report.
    \item Benchmark accuracy across various emotion detection metrics, indicating a relatively functional model.
    \item Improvement in before-and-after lifestyle positivity scores.
\end{itemize}

