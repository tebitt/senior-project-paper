\section{Summary and Real Benefit to Industry}
\subsection{Summary}
In summary, the project focuses on developing an emotional wellbeing robot that promotes mental wellness and positivity for users under the influence of stress and anxiety. The robot will utilize technologies such as computer vision and natural language processing to recognize and respond to emotional cues in real time, providing consistent and empathetic support. With a focus on design, interactivity, and empathy, the robot is tailored to meet the needs of Gen Z and younger millennials in promoting mental wellbeing. The project is structured around two academic semesters, employing an agile methodology to deliver two key prototypes, MVP1 (Proof of Concept) and MVP2 (A Fully Functional Design). The development process is guided by insights from Dr. Paulo Fernando Rocha Garcia, Ph.D., Assistant Professor of AI and Robotics at Chulalongkorn University, and Ms. Kunpariya Siripanit, a counseling psychologist at Chulalongkorn University, ensuring that the robot not only meets technical standards but also aligns with mental health principles. This project offers a scalable, innovative solution that addresses the increasing prevalence of anxiety disorders, positioning the robot as a sustainable alternative to traditional mental health support methods. 
\subsection{Benefits Direct Impact on the Industry}
This project will significantly benefit the mental health field, specifically targeting patients with general anxiety and stress, which comprises a massive customer segment. In particular, mental wellbeing robots are an important initiative in countering \textbf{\textit{Terror Outbursts}} in Thailand. By leveraging AI to detect and respond to emotions through facial expressions and voice analysis, these robots can reduce reliance on human intervention in mental positivity promotion. Additionally, the project will enhance service quality by offering consistent promotion of mental well-being, effectively addressing unmet mental health needs. Scalability and Long-Term Value: With recent studies indicating a 55\% increase in anxiety disorders globally from 1990 to 2019, affecting approximately 301 million people worldwide, these robots will become increasingly essential. Their ability to offer real-time, personalized assistance will not only improve individual well-being but also contribute to the long-term sustainability of mental health care systems. While the project will remain in the proof of concept and prototyping stage, it aims to provide an innovative approach to mental wellbeing, potentially scaling to commercial use in the future. 
