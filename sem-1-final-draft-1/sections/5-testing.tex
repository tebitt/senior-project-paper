\section{Testing}

This section details the proposed approach to testing Pookie on end users. Overall, the testing phase will be conducted in Semester 2 under the supervision of our psychology advisor, Dr. Fah Kunpariya Siripanit. The testing will comprise 3-4 end users diagnosed with general anxiety. These end users will be volunteers provided by Chula Student Wellness, and in some cases, personal connections. 

The selected users will be given Pookie for a total of seven days to incorporate the robot into their daily lives. The rationale for this duration is provided by our advisor, who suggests that seven days is the bare minimum for a user to develop any form of emotional attachment. Once they are given the robot, they will complete a questionnaire both before and after the seven-day period to evaluate whether Pookie has had a positive impact on their daily lives. The questions include self-ratings on current anxiety and positivity levels (on a scale from 1-10). 

Since there is no universal interpretation of these scores, a control group (those without access to Pookie) will also answer the questions, allowing the team to establish baseline scores and draw meaningful insights. The key questions are summarized in Table~\ref{table:questions}. It is important to note that while the general rationale for these questions comes from our advisor, the specific wording and execution of the questionnaire will be finalized next year.

\begin{table}[h]
\centering
\caption{List of Questions for Testing}
\label{table:questions}
\begin{tabular}{|l|l|}
\hline
\textbf{Key Questions} & \textbf{Quantitative Metric} \\ \hline
How would you rate your everyday level of anxiety? & Self-rated scale from 1-10 \\ \hline
How would you rate your maximum level of anxiety? & Self-rated scale from 1-10 \\ \hline
How would you rate your everyday level of positivity? & Self-rated scale from 1-10 \\ \hline
How would you rate your maximum level of positivity? & Self-rated scale from 1-10 \\ \hline
\end{tabular}
\end{table}

In regard to security and PDPA compliance, Pookie will require full consent from the end users for the temporary usage of face and voice data. The team is seeking assistance from Chula Regulations to ensure that Pookie adheres to all necessary legal and ethical requirements for testing.
