\documentclass[a4paper,10pt]{article}

\usepackage{graphicx}
\graphicspath{{sections/images/}}
\usepackage{}
\usepackage{caption}
\usepackage[section]{placeins}
\usepackage{titlesec}

\usepackage{amsmath}
\usepackage{amsfonts}
\usepackage{amssymb}
\usepackage{hyperref}
\usepackage[a4paper, top=2cm, bottom=2cm, left=2.5cm, right=2.5cm]{geometry}

\setcounter{secnumdepth}{4}
\titleformat{\paragraph}
{\normalfont\normalsize\bfseries}{\theparagraph}{1em}{}
\titlespacing*{\paragraph}
{0pt}{3.25ex plus 1ex minus .2ex}{1.5ex plus .2ex}

\begin{document}
\begin{titlepage}
    \centering  
    \begin{figure}[ht]
        \centering
        \includegraphics[width=\textwidth]{ise-logo.png}
    \end{figure}
    \vspace*{2cm} 
    
    {\Huge \textbf{Pookie:} \textit{An AI-Driven Robot for Promoting Mental Wellbeing and Emotional Support} \par}
    \vspace{4cm}
    
    {\large \textbf{Authors:} Tibet Buramarn, Kridbhume Chammanard, and Thitaya Divari \par}
    \vspace{1cm}
    {\large \textbf{Advisor:} Dr. Paulo Fernando Rocha Garcia, Ph.D., Assistant Professor of AI and Robotics at Chulalongkorn University and \par}
    {\large Ms. Kunpariya Siripanit, Counseling Psychologist at Chula Student Wellness, Chulalongkorn University \par}

    \vspace{3cm}
    
    {\large 2147416 Final Project I \par}
    {\large International School of Engineering (ISE) \par}
    {\large Chulalongkorn University \par}
    
    \vspace{2cm}
    
    {\large December 11, 2024 \par}
    
    \vspace*{\fill}
\end{titlepage}

\thispagestyle{empty}

\newpage
\begin{center}
    \item\section*{Abstract}
\end{center}
\large
Pookie, an AI-driven robot for promoting mental wellbeing, developed under advisory with
Chula Student Wellness, aims to create an AI-driven companion to enhance mental wellbeing by promoting positivity. The robot aims to address \textbf{\textit{Terror Outbursts}}, a future
concern in Thailand involving an anxiety-driven society, where the robot aims to alleviate
feelings of stress and anxiety by providing a feeling of slowness and emotional attachment.
The robot integrates computer vision, feature extraction, sensors, and actuators to address
key customer needs in appearance, interactivity, and empathy. The general appearance
features an anthropomorphic form with LED displays and sensors for interaction, drawing
inspiration from existing robots like Kiki, with future iterations expected to
refine user experience and functionality based on feedback.
\newpage
\begin{center}
    \item\section*{Acknowledgements}
\end{center}
The team expresses their sincere gratitude to Dr. Paulo Fernando Rocha Garcia, Ph.D., Assistant Professor of AI and Robotics at Chulalongkorn University, and Ms. Kunpariya Siripanit, Counseling Psychologist at Chula Student Wellness, Chulalongkorn University. Dr. Paulo’s expertise in artificial intelligence and robotics was instrumental in guiding the technical development of the emotional wellness robot, while Ms. Kunpariya’s insights into counseling psychology ensured the project’s alignment with mental health principles. The team also acknowledges the support provided by the faculty and staff of the International School of Engineering, whose assistance was invaluable throughout this project. 
\normalsize

\newpage
\tableofcontents

\newpage
\section{Research Background}
This section will provide justifications for the project and necessary knowledge for the reader in order to
understand technical terms used throughout the report.

\subsection{Justification of the Project}

The project focuses on developing emotional well-being robots that promote mental wellness and positivity for users under the influence of stress and anxiety. In general, emotion-related robots are designed to respond to human emotions and can potentially achieve clinical outcomes similar to traditional therapy \cite{Palmer2024.07.17.24310551}. Research has shown that digital interventions, such as AI-powered mental well-being robots, can effectively reduce anxiety symptoms and address unmet mental health needs, offering a promising solution to supplement traditional therapeutic approaches \cite{jarvis2024companion}.

The mental health industry faces significant challenges that cannot be fully addressed through human intervention alone \cite{charles-2024}. Key issues include loneliness and social isolation, which are major contributors to depression, anxiety, and overall deterioration in mental health \cite{GOH202372}, as well as therapeutic challenges, where patients with dementia or other cognitive impairments often struggle with traditional therapeutic activities \cite{Sukhawathanakul_Crizzle_Tuokko_Naglie_Rapoport_2021}. In a study by the National Innovation Agency (NIA) based in Thailand, it was identified that the concept of \textbf{\textit{Terror Outbursts}} would become a pressing issue in Thailand by the year 2033 \cite{nia2023}. To elaborate, terror outburst refers to a society driven by constant fear and anxiety. Consequently, traditional methods of addressing anxiety, such as therapy and medication, may not be accessible or appealing to everyone. This creates a significant pain point for individuals seeking immediate, non-invasive support. Our target customer segment includes young adults and professionals aged 18-35 who experience mild to moderate anxiety but may be hesitant to seek conventional treatment, where we provide an innovative alternative to support their mental well-being.

To ensure emotional well-being robots meet user needs and deliver effective support, three key pillars are essential: appearance, interactivity, and empathy. First, the robot’s appearance should strike a balance between human-like and machine-like traits, fostering both comfort and trust in users \cite{10.1145/3640794.3665551}. High interactivity is also crucial; the robot should provide adaptive feedback through various stimuli to engage users effectively and enhance their emotional states \cite{Wang_2024}. Moreover, a robot’s perceived empathic abilities play a significant role in how users interact with it, as these perceptions directly influence their willingness to attribute mental states to the robot, thereby impacting the overall quality of the interaction \cite{lillo2024investigatingrelationshipempathyattribution}. On the concept of emotion detection, traditional emotional detection methods utilize verbal and non-verbal cues to accurately detect and respond to human emotions. Verbal cues like pitch variations, volume, and speech rate \cite{HAKANPAA2021570} are critical indicators of emotional states. For example, higher pitch and increased volume often signal heightened emotional arousal, as seen in both American English and Mandarin Chinese, where pitch and speed are essential for expressing emotions. Additionally, contextual understanding—interpreting emotions based on situational cues—further refines the robot’s emotional recognition capabilities \cite{abbas2024context}. Non-verbal cues, such as facial expressions and body language, also play a vital role. For instance, a smile usually denotes happiness, while crossed arms might suggest defensiveness \cite{liu2024emotiondetectionbodygesture}. By integrating these verbal and non-verbal indicators, mental well-being support robots can offer tailored responses, thereby improving the overall effectiveness of their interactions with users.

\subsection{Necessary Knowledge}

The development of a mental well-being support robot with emotional detection capabilities requires a strong foundation in various advanced concepts within artificial intelligence, machine learning, robotics, and human-computer interaction. Below is an overview of the essential knowledge areas for this project:

Machine learning models are the backbone of emotion detection systems. Convolutional Neural Networks (CNNs) \cite{computation11030052} are widely used for tasks such as facial emotion recognition, where they excel at analyzing image data to identify patterns corresponding to different emotional states. One specific architecture, VGGNet, has proven effective for emotion detection due to its deep, layered structure and ability to capture fine-grained facial features. VGGNet's simplicity in design \cite{computation11030052}, using smaller 3x3 filters stacked in depth, makes it particularly useful for recognizing subtle facial expressions that correspond to emotions. This capability enhances the accuracy of emotion detection, which is crucial for the mental well-being support robot to respond appropriately to a user's emotional state.

In the visual domain, key facial features like eyes, mouth, and eyebrows are extracted and analyzed by CNNs to detect emotions from facial expressions. However, effective emotion recognition often requires the consideration of temporal patterns in sequences of images, such as micro-expressions that unfold over time. Recurrent Neural Networks (RNNs) and Long Short-Term Memory (LSTM) \cite{schmidt2019} networks are essential for processing such sequential data. LSTMs, in particular, are highly effective at retaining information over extended time periods, enabling the robot to identify subtle changes in facial expressions or gestures that might otherwise go unnoticed.

Speech recognition is crucial for enabling the robot to understand and interpret human speech, which is key to detecting emotions from spoken input. Speech recognition techniques allow the robot to process audio data, extracting meaningful insights such as tone, pitch, and speech patterns. These insights help the robot assess the emotional tone and context of the user’s communication, making it possible to respond appropriately to their emotional needs. Similar to visual emotion recognition, LSTMs are also invaluable in analyzing sequential audio features, ensuring that variations in tone or pitch over time are captured effectively.

Effective emotion detection relies on extracting meaningful features from raw data. For instance, Mel-Frequency Cepstral Coefficients (MFCCs) \cite{singh2014} are a crucial feature extraction technique in speech emotion recognition, capturing the essential characteristics of the audio signal that correlate with emotional states. Similarly, in the visual domain, CNNs extract and analyze key facial features like eyes, mouth, and eyebrows to detect emotions from facial expressions. The combination of CNNs for spatial analysis and LSTMs for temporal analysis creates a robust framework for identifying emotions from both static and dynamic data.

The design of emotionally intelligent robots requires an understanding of Human-Robot Interaction (HRI) principles. These principles guide the development of robots that can interact naturally and empathetically with humans. Concepts such as user-friendly interface design, adaptive behavior, and empathetic response mechanisms ensure that the robot’s interactions are socially acceptable and supportive.

Additionally, emotionally intelligent robots rely on a combination of advanced hardware and software to accurately detect and respond to human emotions. Key hardware components, including cameras, are essential for capturing detailed facial expressions in real-time, allowing systems to effectively analyze emotional states \cite{gupta-2024}. Microphones and audio sensors play a crucial role in gathering vocal cues, which are vital for emotion detection \cite{10.48175/ijarsct-15385}. Processors and GPUs manage the heavy computational tasks, while actuators and motors control the robot’s physical movements, such as gestures and facial expressions, enabling the robot to convey empathy and respond to users effectively. Haptic sensors further enhance this interaction by reacting to touch, contributing to a more interactive and supportive user experience.

Lastly, Bayesian Networks provide a robust framework for decision-making \cite{DBLP:journals/corr/abs-2002-00269}, enabling the robot to infer emotional states and choose appropriate responses. These graphical models represent variables and their dependencies through directed acyclic graphs (DAGs). For the robot, observable inputs like facial expressions, vocal cues, and contextual data are nodes, while hidden nodes represent inferred emotional states such as sadness or anxiety.

Bayesian Networks allow the robot to integrate prior knowledge and update beliefs with new information. A belief represents the robot's degree of confidence in a particular state or outcome, based on available evidence and prior knowledge. For example, if vocal cues indicate frustration but facial expressions appear neutral, the network can combine these inputs to infer the true emotional state. This approach assists the robot in making informed decisions and avoiding ambiguity or conflicting signals in data.
\newpage    
\section{Objective}
\label{sec:objective}
\subsection{Main Objective Statement}
The primary objective of this project is to design, develop, and deploy an emotional wellness robot capable of recognizing and responding to stress and anxiety symptoms in users through the integration of AI technologies such as facial emotion recognition and speech emotion recognition. The robot must fulfill all three key pillars of customer expectations in emotional wellness robots: design, interactivity, and empathy.


\subsection{Specific Goals}
\begin{itemize}
    \item Design intuitive appearance and interactive features for the robot with expert supervision from Chula Student Wellness (CUSW).
    \item Develop an accurate emotion detection algorithm that captures the user’s emotional state, leveraging software design principles taught throughout the curriculum.
    \item Develop empathetic human-robot interactions that promote emotional wellness within the customer, leveraging various engineering design principles taught throughout the curriculum.
    \item Conduct extensive testing and refinement based on user feedback.
\end{itemize}
    
\subsection{Measurable Outcomes}
\begin{itemize}
    \item Achieve a relative reduction in self-reported anxiety among tested users.
    \item Achieve a benchmark in accuracy metrics for emotion detection.
    \item Obtain an improved before-and-after positivity score among tested users.
\end{itemize}

\subsection{Relevance or Significance}
With “Terror Outbursts” being one of the major societal challenges in Thailand, there is a pressing need
for accessible positivity promotion. Our robot aims to bridge the gap between traditional therapy sessions
by providing immediate support to individuals struggling with anxiety

\newpage
\section{Literature Survey and Review}
This section covers the literature survey related to the content defined in the objectives section. It is important to note that this survey encompasses only additional literature that was not initially covered in the first project proposal.

\subsection{Related Works: Existing Products and Technologies}

\begin{enumerate}
    \item{\bf{ElliQ}}
    \vspace{0.25cm}


ElliQ, developed by Intuition Robotics, is an AI-driven social robot designed to address loneliness and promote well-being in older adults. ElliQ is a proactive and conversational companion that facilitates engagement through voice interaction, touch-screen activities, music, video calls, and cognitive games. The robot's primary goal is to reduce social isolation by fostering meaningful interactions and promoting an active lifestyle. ElliQ features a sleek, immobile tabletop design with an expressive lamp-like head that swivels to indicate engagement. Using proprietary AI algorithms, the robot autonomously initiates and personalizes suggestions based on the user’s learned behaviors, sentiment analysis, and past interactions. Over time, the AI adapts its interactions to align with the user’s preferences and routines, fostering a sense of companionship and trust.

ElliQ has demonstrated significant potential in reducing loneliness and improving emotional well-being. Studies conducted in collaboration with healthcare organizations, including the New York State Office for the Aging (NYSOFA), revealed that 80\% of users reported feeling less lonely with ElliQ, while 74\% noted an improvement in their overall quality of life. These findings highlight the effectiveness of social robots as emotional support tools, providing daily engagement, mental stimulation, and social connection. Unlike traditional loneliness interventions that require human facilitators, ElliQ's autonomous nature allows for scalable deployment. By proactively initiating interactions, the robot encourages users to engage in activities that promote mental health, such as guided mindfulness exercises, cognitive challenges, and storytelling. Additionally, ElliQ supports social connection by facilitating video calls with family members, further reinforcing its role as a social catalyst.

Despite its promising impact, ElliQ faces certain challenges, including user hesitation in accepting robotic companionship, technical support requirements, and limitations in conversational fluidity compared to human interactions. Additionally, further research is needed to evaluate its long-term effects on mental health and well-being. Future iterations of ElliQ may integrate more advanced AI-driven conversational capabilities, greater mobility, and enhanced customization features to further optimize user experience. The success of ElliQ underscores the broader potential of social robots in addressing emotional well-being beyond elderly care. Its application can extend to individuals experiencing social isolation due to disability, remote work environments, or other circumstances where human interaction is limited. As AI technology continues to advance, social robots like ElliQ will play an increasingly vital role in promoting mental wellness and emotional resilience \cite{Broadbent2024}.


\begin{figure}[ht]
    \centering
    \includegraphics[width=0.6\textwidth]{elliq.png}
    \caption{ElliQ, Source: Adapted from \cite{ieee2023elliq}}
    \label{fig:elliq}
\end{figure}

\newpage
    \item{\bf{LOVOT}}
    \vspace{0.25cm}


LOVOT, developed by Groove X in Japan, is a social robot designed to provide companionship, particularly for older adults experiencing loneliness. Unlike stationary robots, LOVOT features a mobile, pet-like design equipped with AI-driven learning capabilities, allowing it to recognize users, respond to touch, and engage in affectionate interactions. The robot's design incorporates emotional expressiveness, including eye contact, physical warmth, and responsive movement, making it an appealing alternative to traditional social companionship.

A study by Tan et al. \cite{tan2024lovot} examined the impact of LOVOT on single older adults’ social well-being. Participants in the study interacted with LOVOT independently in their homes over a week and later shared their experiences in interviews. The study identified four key themes from these interactions: caring for the social robot, finding companionship, forming meaningful connections, and comparing the robot with traditional pets. Users reported that LOVOT provided emotional comfort and reduced feelings of loneliness, reinforcing the idea that social robots can serve as viable companions for older adults who live alone. Additionally, the participants expressed a preference for LOVOT over pets due to its lower maintenance requirements and its ability to provide companionship without the need for feeding or grooming.

LOVOT's adaptive AI enables it to tailor its behavior to individual users, reinforcing a sense of personal connection. This ability to form unique interactions based on user behavior sets LOVOT apart from other social robots. The study further emphasized the importance of designing robots that foster meaningful social engagement rather than serving as mere technological novelties. These findings suggest that social robots like LOVOT could play a crucial role in mitigating loneliness among aging populations and improving overall well-being.
\end{enumerate}


\begin{figure}[ht]
    \centering
    \includegraphics[width=0.6\textwidth]{lovot.png}
    \caption{LOVOT, Source: Adapted from \cite{lovot2024}}
    \label{fig:lovot}
\end{figure}

\subsection{State of the Art}

Bayesian Networks (BNs) are a proven tool for decision-making under uncertainty, as demonstrated by Rothmund et al. \cite{rothmund2021bayesian}, where Dynamic Bayesian Networks (DBNs) were applied to enhance the autonomy of industrial drones. The drones utilized DBNs to infer internal faults, assess environmental conditions, and make proactive decisions to avoid failures while executing independent tasks. By integrating information over time and dynamically updating beliefs, the drones optimized task execution while minimizing risks and the consequences of failures. Our project draws upon similar principles to develop a mental wellness robot designed to promote positivity and reduce stress. Although our robot operates with predefined action bubbles, which are structured interactions tailored to various user emotions, Bayesian Networks can play a vital role in determining which action bubble to deploy based on the user’s current emotional state.

Similar to the drone’s ability to assess environmental conditions, our robot can use a Bayesian Network to infer a user’s emotional state from multiple observable inputs such as facial expressions, vocal tone, and speech patterns. For instance, if vocal cues suggest frustration while facial expressions indicate neutrality, the Bayesian model can combine these observations to probabilistically identify the user’s dominant emotional state. Following the decision-making approach in the drone study, the Bayesian Network can evaluate which predefined action bubble, such as a greeting, playful movement, or verbal feedback, would most effectively promote wellness in the user. By assessing probabilistic relationships between input signals and predefined user emotions, the system ensures that interactions feel relevant and positive.

Emotional indicators are often incomplete or conflicting, such as vocal tone indicating stress while facial expressions suggest calmness. The Bayesian framework excels in such scenarios by integrating prior knowledge and real-time evidence to make confident decisions, ensuring that the robot’s interactions remain meaningful and appropriate. Rothmund et al. \cite{rothmund2021bayesian} emphasized the importance of minimizing risks in decision-making. Similarly, our robot uses Bayesian methods to weigh the likelihood of success for various action bubbles. For example, if the evidence suggests high uncertainty in emotional detection, the robot can select neutral or universally positive interactions to avoid a mismatch between the user’s needs and the robot’s response. The inclusion of Bayesian Networks ensures that the mental wellness robot adapts dynamically to user states, optimizing its responses to enhance emotional support and well-being.
\newpage
\section{Project Concept Development}

\subsection{Acknowledgement}
The project concept was developed under consultation with Ms. Kunpariya Siripanit, Counseling Psychologist at Chula Student Wellness (CUSW), Chulalongkorn University.

\subsection{Project Overview}
\textbf{Pookie}, the emotional well-being robot, is conceptualized as a responsive, AI-driven companion designed to improve mental well-being, specifically catered to customers under stress and anxiety. In this project, \textbf{anxiety} is indicated by signs of stress or worry, and \textbf{emotional well-being} is defined as \textit{"positivity promotion."}

The project focuses on \textbf{promotion}, meaning the promotion of positive well-being through the use of robotics, rather than \textbf{prevention}, which refers to the goal of preventing long-term issues such as depression or suicide.

The robot is envisioned to satisfy three key pillars of customer needs:
\begin{itemize}
    \item \textbf{Appearance}
    \item \textbf{Interactivity}
    \item \textbf{Empathy}
\end{itemize}
These will be achieved through the intuitive integration of computer vision, feature extraction, sensors, and actuators.

\subsection{Limitations and Scope}
This project aims to develop a proof of concept for an emotional well-being robot. However, several limitations and scope considerations apply:
\begin{enumerate}
    \item\textbf{Security:}
        \begin{itemize}
            \item The focus is on creating a prototype that demonstrates the feasibility of an emotional well-being robot.
            \item Security measures will be basic, and advanced features like data encryption and user authentication are beyond the project’s scope.
        \end{itemize}
    \item\textbf{Safety:}
        \begin{itemize}
            \item Fundamental safety, including electronics, heat output, and physical design, will be ensured through rigorous testing.
            \item Detailed safety protocols, such as long-term durability and fail-safes for unforeseen hazards, are outside the scope of this prototype phase.
        \end{itemize}
    \item\textbf{Functionality:}
        \begin{itemize}
            \item The robot will provide core emotional well-being functionalities such as basic interaction and mood assessment.
            \item Advanced features, like personalized therapeutic interventions or integration with external health systems, will not be included in this prototype.
        \end{itemize}
    \item\textbf{User Experience:}
        \begin{itemize}
            \item The prototype will offer a foundational user experience, but may lack the polish and customization of fully developed models.
            \item User interface enhancements will be addressed in future, scaled development phases.
        \end{itemize}
    \item\textbf{Scalability:}
        \begin{itemize}
            \item The project will not focus on scalability for mass production or widespread deployment.
            \item The prototype demonstrates initial concepts and feasibility, not full-scale implementation.
        \end{itemize}
    \item\textbf{Integration:}
        \begin{itemize}
            \item Extensive integration with other technologies or platforms is not within the project's focus.
            \item Emphasis will be on the standalone capabilities of the robot, with minimal focus on interoperability with existing systems.
        \end{itemize}
\end{enumerate}

By acknowledging these limitations and scope considerations, this project sets clear expectations for its initial development phase. Future iterations may address these areas in greater detail based on feedback and further research.

\newpage
\subsection{Hardware Implementation}
This section details the hardware implementation for Pookie. Overall, the design consists of a robotic shell housing an array of electronic components including the microprocessor, motors, sensors, and so on. A fully functional hardware implementation was not within the scope for this semester, which focused on a basic software proof of concept, and physical design.


\subsubsection{Initial Design}
\begin{enumerate}

    \item\textbf{Outer Shell Design}
    
    In the initial phase, a preliminary design is developed for the outer shell. The design is inspired by Rilakkuma, a well-known Japanese character created by San-X in 2003. Rilakkuma is recognized for its comforting and cute aesthetic, which represents relaxation and embodies the essence of kawaii culture. Its rounded features and laid-back attitude have made it a popular symbol of comfort and appeal \cite{hinka_rilakkuma_history}.
    
    \begin{figure}[ht]
        \centering
        \includegraphics[width=0.8\textwidth]{init-outer.png}
        \caption{Initial Outer Shell Design of Pookie}
        \label{fig:init-outer}
    \end{figure}

    \item\textbf{Inner Shell Mechanism Design}

    The arms are designed to move along the x-axis revolute joint and are powered by CS002 20 Kg servo motors.
    \begin{figure}[ht]
        \centering
        \includegraphics[width=0.7\textwidth]{inner-arm.png}
        \caption{Initial Arm Mechanism of Pookie}
        \label{fig:inner-arm}
    \end{figure}
    
\end{enumerate}
\subsubsection{Revised Design}
The revised hardware design of the Pookie robot focuses on anthropomorphic aesthetics, user-centric functionality, seamless mechanical integration, changes in the arm movement axis, and a more compact size.

\begin{figure}[ht]
    \centering
    \includegraphics[width=0.8\textwidth]{revised.png}
    \caption{Revised Design of Pookie}
    \label{fig:revised}
\end{figure}

\begin{enumerate}
    \item\textbf{Outer Shell Design}

    The outer shell of Pookie is designed to enclose all internal components in a compact form. It consists of two main sections: the body and the head. The body houses the internal systems, while the head contains an LED screen for displaying the robot's eyes and serves as the mounting point for the head movement mechanism.

\begin{figure}[ht]
    \centering
    \includegraphics[width=0.8\textwidth]{new_outer.png}
    \caption{Revised Outer Shell Body of Pookie}
    \label{fig:revised_outer}
\end{figure}

\begin{figure}[ht]
    \centering
    \includegraphics[width=0.7\textwidth]{new_head.png}
    \caption{Revised Outer Shell Head of Pookie}
    \label{fig:revised_head}
\end{figure}

\newpage
\item\textbf{Inner Shell Mechanism Design}

The movement mechanisms in Pookie include systems for the head, arms, and base. Each actuator is represented as a cylinder, indicating rotational motion driven by MG90s servo motors. The motion occurs along defined axes, with the z-axis primarily supporting rotational movements. 

\begin{figure}[!htb]
    \centering
    \includegraphics[width=0.7\textwidth]{motion.png}
    \caption{Motion Axes for Pookie's Movement Mechanisms}
    \label{fig:motion}
\end{figure}

Both the head and arm movements use a bevel gear mechanism with a 1:1 transmission powered by MG90s servo motors. This system provides a single degree of freedom along the \textbf{y-axis} revolute joint. A magnet slot at the tip allows easy attachment to the outer shell head, simplifying assembly and maintenance.

\begin{figure}[!htb]
    \centering
    \includegraphics[width=0.7\textwidth]{gear_diagram.png}
    \caption{Components of the Head and Arm Mechanisms}
    \label{fig:components_head_arm}
\end{figure}

The bevel gear mechanism transfers motion from the servo motor to the movement axis efficiently. With a 1:1 transmission ratio, the angular velocity (\(\omega\)) of the driving gear is equal to that of the driven gear. This relationship is expressed as:

\[
\omega_{\text{output}} = \omega_{\text{input}}
\]

where \(\omega_{\text{input}}\) is the angular velocity of the motor, and \(\omega_{\text{output}}\) is the angular velocity of the output gear. Similarly, the angular acceleration (\(\alpha\)) remains unchanged during transmission:

\[
\alpha_{\text{output}} = \alpha_{\text{input}}
\]

This design ensures that the servo motor’s precise control over movement translates directly to the head and arm mechanisms. The torque transmitted by the system is given by:

\[
\tau_{\text{output}} = \tau_{\text{input}}
\]

where \(\tau_{\text{input}}\) is the torque generated by the servo motor. The selection of a 1:1 transmission ratio supports direct coupling, allowing for seamless motion transmission without the need for additional adjustments or gear reduction.

For the base rotation, a bearing-supported system allows full 360-degree rotation along the z-axis. This system incorporates a low-friction bearing assembly, ensuring smooth and precise rotational motion.

\begin{figure}[!htb]
    \centering
    \includegraphics[width=0.7\textwidth]{new_base.png}
    \caption{Components of the Base Mechanism}
    \label{fig:base_comp}
\end{figure}

An armor has been designed to house and support the head, arm, and base rotation mechanisms. It includes compartments for the servo motor and the PWM servo motor driver, as well as openings to facilitate wiring integration. This armor ensures structural integrity and maintains the functionality of all movement mechanisms within Pookie’s compact design.

\begin{figure}[!htb]
    \centering
    \includegraphics[width=0.7\textwidth]{new_armor.png}
    \caption{Internal Armor Design of Pookie}
    \label{fig:new_armor}
\end{figure}


\end{enumerate}


\newpage
\section{Testing}

This section details the proposed approach to testing Pookie on end users. Overall, the testing phase will be conducted in Semester 2 under the supervision of our psychology advisor, Dr. Fah Kunpariya Siripanit. The testing will comprise 3-4 end users diagnosed with general anxiety. These end users will be volunteers provided by Chula Student Wellness, and in some cases, personal connections. 

The selected users will be given Pookie for a total of seven days to incorporate the robot into their daily lives. The rationale for this duration is provided by our advisor, who suggests that seven days is the bare minimum for a user to develop any form of emotional attachment. Once they are given the robot, they will complete a questionnaire both before and after the seven-day period to evaluate whether Pookie has had a positive impact on their daily lives. The questions include self-ratings on current anxiety and positivity levels (on a scale from 1-10). 

Since there is no universal interpretation of these scores, a control group (those without access to Pookie) will also answer the questions, allowing the team to establish baseline scores and draw meaningful insights. The key questions are summarized in Table~\ref{table:questions}. It is important to note that while the general rationale for these questions comes from our advisor, the specific wording and execution of the questionnaire will be finalized next year.

\begin{table}[h]
\centering
\caption{List of Questions for Testing}
\label{table:questions}
\begin{tabular}{|l|l|}
\hline
\textbf{Key Questions} & \textbf{Quantitative Metric} \\ \hline
How would you rate your everyday level of anxiety? & Self-rated scale from 1-10 \\ \hline
How would you rate your maximum level of anxiety? & Self-rated scale from 1-10 \\ \hline
How would you rate your everyday level of positivity? & Self-rated scale from 1-10 \\ \hline
How would you rate your maximum level of positivity? & Self-rated scale from 1-10 \\ \hline
\end{tabular}
\end{table}

In regard to security and PDPA compliance, Pookie will require full consent from the end users for the temporary usage of face and voice data. The team is seeking assistance from Chula Regulations to ensure that Pookie adheres to all necessary legal and ethical requirements for testing.

\newpage
\section{Summary and Real Benefit to Industry}
\subsection{Summary}
In summary, the project focuses on developing an emotional wellbeing robot that promotes mental wellness and positivity for users under the influence of stress and anxiety. The robot will utilize technologies such as computer vision and natural language processing to recognize and respond to emotional cues in real time, providing consistent and empathetic support. With a focus on design, interactivity, and empathy, the robot is tailored to meet the needs of Gen Z and younger millennials in promoting mental wellbeing. The project is structured around two academic semesters, employing an agile methodology to deliver two key prototypes, MVP1 (Proof of Concept) and MVP2 (A Fully Functional Design). The development process is guided by insights from Dr. Paulo Fernando Rocha Garcia, Ph.D., Assistant Professor of AI and Robotics at Chulalongkorn University, and Ms. Kunpariya Siripanit, a counseling psychologist at Chulalongkorn University, ensuring that the robot not only meets technical standards but also aligns with mental health principles. This project offers a scalable, innovative solution that addresses the increasing prevalence of anxiety disorders, positioning the robot as a sustainable alternative to traditional mental health support methods. 
\subsection{Benefits Direct Impact on the Industry}
This project will significantly benefit the mental health field, specifically targeting patients with general anxiety and stress, which comprises a massive customer segment. In particular, mental wellbeing robots are an important initiative in countering \textbf{\textit{Terror Outbursts}} in Thailand. By leveraging AI to detect and respond to emotions through facial expressions and voice analysis, these robots can reduce reliance on human intervention in mental positivity promotion. Additionally, the project will enhance service quality by offering consistent promotion of mental well-being, effectively addressing unmet mental health needs. Scalability and Long-Term Value: With recent studies indicating a 55\% increase in anxiety disorders globally from 1990 to 2019, affecting approximately 301 million people worldwide, these robots will become increasingly essential. Their ability to offer real-time, personalized assistance will not only improve individual well-being but also contribute to the long-term sustainability of mental health care systems. While the project will remain in the proof of concept and prototyping stage, it aims to provide an innovative approach to mental wellbeing, potentially scaling to commercial use in the future. 

\newpage
\input{sections/7-workload}

\newpage
\addcontentsline{toc}{section}{References}
\bibliographystyle{IEEEtran}
\bibliography{sections/references}

\end{document}
